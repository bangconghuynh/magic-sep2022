\section{Unitary representation analysis}

  \subsection{General formulation}
  %%%%%%%%%%%%%%%%%%%%%%%%%%
  %%%%%%%%%%%%%%%%%%%%%%%%%%
  \begin{frame}{Group unitary representations on linear spaces}
    \begin{itemize}
      \item<1-> Consider a \emphbf{DarkGreen}{group $\gls*{struct:gengroup}$} that acts \emphit{Magenta}{unitarily} on a \emphbf{DarkGreen}{linear space $V$}.

      \item<1-> Let $\symbfit{v}$ be an element in $V$.
      The unitary action of $\gls*{struct:gengroup}$ on $\symbfit{v}$ generates an \emphbf{DarkGreen}{orbit}
      \begin{equation*}
        \gls*{struct:gengroup} \cdot \symbfit{v}
          = \{ \hat{u}_i \symbfit{v} \mid u_i \in \gls*{struct:gengroup} \}
      \end{equation*}
      which spans a \emphbf{DarkGreen}{representation subspace} $\textcolor{DarkGreen}{\gls*{struct:rep}} \subseteq V$.\\
      {\footnotesize For simplicity, we will assume that $\gls*{struct:gengroup} \cdot \symbfit{v}$ is a linearly independent basis.}

      \item<2-> We seek to decompose $\gls*{struct:rep}$ into known \emphit{Magenta}{irreducible} representations of $\gls*{struct:gengroup}$ on $V$.\\[6pt]
        $\hookrightarrow$ This \emphit{Magenta}{quantifies} the \emphit{Magenta}{symmetry} of $\symbfit{v}$ under the action of $\gls*{struct:gengroup}$.

      \item<2-> To this end, we pick a reference element $\symbfit{v}_i = \hat{u}_i \symbfit{v}$ in $\gls*{struct:gengroup} \cdot \symbfit{v}$ and define the \emphbf{DarkGreen}{representation matrices $\symbfit{D}^{\gls*{struct:rep}}(u_k)$} for all $u_k \in \gls*{struct:gengroup}$:
      \begin{equation*}
        \hat{u}_k \symbfit{v}_i = \sum_{j} \symbfit{v}_j D^{\gls*{struct:rep}}_{ji}(u_k).
      \end{equation*}
      Their traces, \emphbf{DarkGreen}{$\chi^{\gls*{struct:rep}}(u_k)$}, give the \emphbf{DarkGreen}{characters} required for the decomposition of $\gls*{struct:rep}$.
    \end{itemize}
  \end{frame}


  %%%%%%%%%%%%%%%%%%%%%%%%%%
  %%%%%%%%%%%%%%%%%%%%%%%%%%
  \begin{frame}{Representation matrix determination}
    \begin{itemize}
      \item<1-> We opt for a projection method.\\
      $\hookrightarrow$ This requires $V$ to be \emphit{Magenta}{endowed} with an \emphbf{DarkGreen}{inner product $\braket{\cdot | \cdot}$}.

      \item<2-> Define a non-orthogonal projection operator
      \begin{equation*}
        \hat{\symscr{P}}_i =
          \sum_j \ket{\symbfit{v}_i} (\symbfit{S}^{-1})_{ij} \bra{\symbfit{v}_j}
      \end{equation*}
      where $S_{ij} = \braket{\symbfit{v}_i | \symbfit{v}_j}$ such that $\hat{\symscr{P}}_i \ket{\symbfit{v}_j} = \delta_{ij} \ket{\symbfit{v}_i}$.

      \item<3-> Application of $\hat{\symscr{P}}$ on the defining equation for $\symbfit{D}^{\gls*{struct:rep}}(u_k)$ yields
      \begin{equation*}
        \symbfit{D}^{\gls*{struct:rep}}(u_k) = \symbfit{S}^{-1} \symbfit{T}(u_k),
      \end{equation*}
      where
      \begin{equation*}
        \tikzmarkin<4>[nodeStyleGreen, mark at=0.56]{Tijk}(0.04,-0.20)(-0.06,0.40)
          \annotate{4}{red}{(0,0)--++(0,-0.3)}{left}{%
            $\lvert \gls*{struct:gengroup} \rvert^3$ elements%
          }
          T_{ij}(u_k)
        \tikzmarkend{Tijk}
          = \braket{\symbfit{v}_i | \hat{u}_k \symbfit{v}_j}.
      \end{equation*}

      \item<4-> \emphit{Magenta}{Closure} of $\gls*{struct:gengroup}$ $\implies$ $T_{ij}(u_k)$ can be \emphit{Magenta}{mapped to $T_{m1}(e)$} for some $m = 1, \ldots, \lvert \gls*{struct:gengroup} \rvert$.\\
      $\hookrightarrow$ $\symbfit{T}(u_k)$ can be computed with \emphit{Magenta}{$\symcal{O}(\lvert \gls*{struct:gengroup} \rvert)$} time complexity.
    \end{itemize}

    \footlessfullcite{article:Soriano2014}
  \end{frame}


  \subsection{Formulation for current densities}
  %%%%%%%%%%%%%%%%%%%%%%%%%%
  %%%%%%%%%%%%%%%%%%%%%%%%%%
  \begin{frame}{Current density linear space}
    \begin{itemize}
      \item<1-> The current densities $\gls*{mag:jphys}(\gls*{bas:spatialcoord})$ with $\gls*{bas:spatialcoord} \in \symbb{R}^3$ form a linear space $V_J$.
      \item<1-> Define an inner product $\braket{\cdot | \cdot}$ on $V_J$ as
      \begin{equation*}
        \braket{\gls*{mag:jphys}_m | \gls*{mag:jphys}_n}
        = \int
          \gls*{mag:jphys}_m(\gls*{bas:spatialcoord})^{\dagger}
          \ \gls*{mag:jphys}_n(\gls*{bas:spatialcoord})
          \ \D\gls*{bas:spatialcoord}.
      \end{equation*}
      $\hookrightarrow$ Enables projection-based unitary representation analysis on $V_J$
      \item<2-> Given a current density $\gls*{mag:jphys}(\gls*{bas:spatialcoord})$ and a symmetry or pseudo-symmetry \emphit{Magenta}{unitary} group $\gls*{struct:gengroup}$, the required overlap matrix elements for the symmetry analysis of $\gls*{mag:jphys}(\gls*{bas:spatialcoord})$ are of the form
      \begin{equation*}
        T_{m1}(e)
          = \braket{\hat{u}_m \gls*{mag:jphys} | \gls*{mag:jphys}}
          = \int
            \tikzmarkin<3>[nodeStyleGreen, mark at=0.82]{umj}(0.04,-0.20)(-0.06,0.40)
              \annotate{3}{red}{(0,0)--++(0,-0.3)}{below}{%
                (pseudo-)symmetry-transformed current density%
              }
              (\hat{u}_m \gls*{mag:jphys})
            \tikzmarkend{umj}(\gls*{bas:spatialcoord})^{\dagger}
            \ \gls*{mag:jphys}(\gls*{bas:spatialcoord})
            \ \D\gls*{bas:spatialcoord},
        \quad
        u_m \in \gls*{struct:gengroup}.
      \end{equation*}
    \end{itemize}
  \end{frame}


  %%%%%%%%%%%%%%%%%%%%%%%%%%
  %%%%%%%%%%%%%%%%%%%%%%%%%%
  \begin{frame}{Non-perturbative current densities}
    \begin{itemize}
      \item<1-> Non-perturbative calculations in arbitrarily strong magnetic fields are performed in a basis of \emphbf{DarkGreen}{London \glsxtrlongpl{acr:ao}}:
      \begin{equation*}
        \gls*{bas:londonspatialbasis}[_{\mu}](\gls*{bas:spatialcoord}; \symbfit{R}_{\mu}) =
          \gls*{bas:spatialbasis}[_{\mu}](\gls*{bas:spatialcoord}; \symbfit{R}_{\mu})
          \exp\left[-i \gls*{mag:vecpot}(\symbfit{R}_{\mu}) \cdot \gls*{bas:spatialcoord}\right].
      \end{equation*}

      \item<1-> In this basis, the current density can be partitioned into the \emphbf{blue}{diamagnetic} and \emphbf{red}{paramagnetic} contributions with the \emphit{Magenta}{non-perturbative} forms:
      \begin{gather*}
        \gls*{mag:jphys}(\gls*{bas:spatialcoord}) = \textcolor{blue}{\gls*{mag:jd}(\gls*{bas:spatialcoord})} + \textcolor{red}{\gls*{mag:jp}(\gls*{bas:spatialcoord})}\\
        \textcolor{blue}{\gls*{mag:jd}(\gls*{bas:spatialcoord})}
        = -\gls*{mag:vecpot}(\gls*{bas:spatialcoord})
          \sum_{\sigma} \gls*{bas:londonspatialbasis}[^*_{\mu}](\gls*{bas:spatialcoord})
          \gls*{bas:londonspatialbasis}[_{\nu}](\gls*{bas:spatialcoord})
          P_{\sigma}^{\nu\mu},
        \qquad
        \textcolor{red}{\gls*{mag:jp}(\gls*{bas:spatialcoord})}
        = \frac{i}{2}
          \sum_{\sigma} \symbf{\nabla}\gls*{bas:londonspatialbasis}[^*_{\mu}](\gls*{bas:spatialcoord})
          \gls*{bas:londonspatialbasis}[_{\nu}](\gls*{bas:spatialcoord})
          P_{\sigma}^{\nu\mu} + \textrm{c.c.}
      \end{gather*}

      \item<2-> This partition depends on the gauge origin $\symbfit{G}$ which manifests itself in
      \begin{equation*}
        \gls*{mag:vecpot}(\gls*{bas:spatialcoord}) =
          \frac{1}{2} \gls*{mag:vec} \times (\gls*{bas:spatialcoord} - \symbfit{G}).
      \end{equation*}
    \end{itemize}

    \footlessfullcite{article:Soncini2004}
    \footlessfullcite{article:Tellgren2008,article:Tellgren2014}
    \footlessfullcite{article:Irons2021}
  \end{frame}


  %%%%%%%%%%%%%%%%%%%%%%%%%%
  %%%%%%%%%%%%%%%%%%%%%%%%%%
  \begin{frame}{Ipsocentric \textit{DZ}}
    \begin{itemize}
      \item<1-2> We employ the \emphbf{DarkGreen}{ipsocentric \textit{DZ}} method which makes use of a continuous set of gauge transformations:
      \begin{equation*}
        \symbfit{G} \equiv \symbfit{G}(\gls*{bas:spatialcoord}) = \gls*{bas:spatialcoord},
      \end{equation*}
      so that $\textcolor{blue}{\gls*{mag:jd}(\gls*{bas:spatialcoord})}$ vanishes and $\gls*{mag:jphys}(\gls*{bas:spatialcoord}) = \textcolor{red}{\gls*{mag:jp}(\gls*{bas:spatialcoord})}$ formally.

      \item<2> In addition,
      \begin{align*}
        \gls*{bas:londonspatialbasis}[_{\mu}](\gls*{bas:spatialcoord}; \symbfit{R}_{\mu})
        &= \gls*{bas:spatialbasis}[_{\mu}](\gls*{bas:spatialcoord}; \symbfit{R}_{\mu})
          \exp\left[-\frac{i}{2} (\gls*{mag:vec} \times (\symbfit{R}_{\mu} - \textcolor{DarkGreen}{\gls*{bas:spatialcoord}})) \cdot \gls*{bas:spatialcoord}\right]\\
        &= \gls*{bas:spatialbasis}[_{\mu}](\gls*{bas:spatialcoord}; \symbfit{R}_{\mu})
          \exp\left[-\frac{i}{2} (\gls*{mag:vec} \times \symbfit{R}_{\mu}) \cdot \gls*{bas:spatialcoord}\right],
      \end{align*}
      which is the same as keeping $\symbfit{G}$ at the space-fixed origin.
    \end{itemize}

    \footlessfullcite{article:Keith1993a}
    \footlessfullcite{article:Soncini2004}
  \end{frame}


  %%%%%%%%%%%%%%%%%%%%%%%%%%
  %%%%%%%%%%%%%%%%%%%%%%%%%%
  \begin{frame}{Integrals}
    \begin{itemize}
      \item<1-> The required overlap matrix elements for the symmetry analysis of $\gls*{mag:jphys}(\gls*{bas:spatialcoord})$ may now be cast in a computable form:
        \begin{align*}
          T_{m1}(e)
            &= \int
              (\hat{u}_m \textcolor{red}{\gls*{mag:jp}})(\gls*{bas:spatialcoord})^{\dagger}
              \ \textcolor{red}{\gls*{mag:jp}}(\gls*{bas:spatialcoord})
              \ \D\gls*{bas:spatialcoord}\\
            &= \begin{multlined}[t]
              \frac{1}{4}
                \sum_{\sigma\sigma'}
                  (P^{\nu\mu}_{\sigma})^*
                  P^{\nu'\mu'}_{\sigma'}\\
                  \tikzmarkin<2>[nodeStyleGreen, mark at=0.77]{2d4cint}(0.04,-0.35)(-0.06,0.55)
                    \annotate{2}{red}{(0,0)--++(0,-0.3)}{below}{%
                      second derivatives of four-centre overlap integrals%
                    }
                    \int
                      (\hat{u}_m \gls*{bas:londonspatialbasis}[^*_{\nu}])(\gls*{bas:spatialcoord})
                      \ \symbf{\nabla}
                      \tikzmarkin<3>[nodeStyleGreen, mark at=0.81]{umj}(0.04,-0.20)(-0.06,0.40)
                        \annotate{3}{red}{(0,0)--++(0,-0.3)}{below}{%
                          (pseudo-)symmetry-transformed London orbitals%
                        }
                        (\hat{u}_m \gls*{bas:londonspatialbasis}[_{\mu}])
                      \tikzmarkend{umj}(\gls*{bas:spatialcoord})^{\T}
                      \left[
                        \symbf{\nabla}
                        \gls*{bas:londonspatialbasis}[^*_{\mu'}](\gls*{bas:spatialcoord})
                        \ \gls*{bas:londonspatialbasis}[_{\nu'}](\gls*{bas:spatialcoord})
                        -
                        \symbf{\nabla}
                        \gls*{bas:londonspatialbasis}[_{\mu'}](\gls*{bas:spatialcoord})
                        \ \gls*{bas:londonspatialbasis}[^*_{\nu'}](\gls*{bas:spatialcoord})
                      \right]
                      \ \D \gls*{bas:spatialcoord}
                  \tikzmarkend{2d4cint}
                  + \textrm{c.c.}
              \end{multlined}
        \end{align*}
    \end{itemize}
  \end{frame}