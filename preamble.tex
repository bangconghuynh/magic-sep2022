% ========
% Preamble
% ========

% -----
% Class
% -----
\documentclass{bredelebeamer}

% ------------
% Slide layout
% ------------
\setbeamersize{%
  text margin left=0.3cm,%
  text margin right=0.3cm
}
%\usepackage{ragged2e}

\usepackage{appendixnumberbeamer}

% List spacing (enumitem is not compatible with beamer)
% https://jayrobwilliams.com/posts/2019/10/better-beamer
% https://tex.stackexchange.com/questions/369504#369597
\makeatletter
\renewcommand{\itemize}[1][]{%
  \beamer@ifempty{#1}{}{\def\beamer@defaultospec{#1}}%
  \ifnum \@itemdepth >2\relax\@toodeep\else
    \advance\@itemdepth\@ne
    \beamer@computepref\@itemdepth% sets \beameritemnestingprefix
    \usebeamerfont{itemize/enumerate \beameritemnestingprefix body}%
    \usebeamercolor[fg]{itemize/enumerate \beameritemnestingprefix body}%
    \usebeamertemplate{itemize/enumerate \beameritemnestingprefix body begin}%
    \list
    {\usebeamertemplate{itemize \beameritemnestingprefix item}}
    {\def\makelabel##1{%
        {%
          \hss\llap{{%
              \usebeamerfont*{itemize \beameritemnestingprefix item}%
              \usebeamercolor[fg]{itemize \beameritemnestingprefix item}##1}}%
        }%
      }%
    }
  \fi%
  \setlength\itemsep{\fill}
  \ifnum \@itemdepth >1
    \vfill
  \fi%
  \beamer@cramped%
  \raggedright%
  \beamer@firstlineitemizeunskip%
}
\def\enditemize{\ifhmode\unskip\fi\endlist%
  \usebeamertemplate{itemize/enumerate \beameritemnestingprefix body end}
  \ifnum \@itemdepth >1
  \vfil
  \fi%
}
\makeatother

\makeatletter
\def\enumerate{%
  \ifnum\@enumdepth>2\relax\@toodeep
  \else%
  \advance\@enumdepth\@ne%
  \edef\@enumctr{enum\romannumeral\the\@enumdepth}%
  \advance\@itemdepth\@ne%
  \fi%
  \beamer@computepref\@enumdepth% sets \beameritemnestingprefix
  \edef\beamer@enumtempl{enumerate \beameritemnestingprefix item}%
  \@ifnextchar[{\beamer@@enum@}{\beamer@enum@}}
\def\beamer@@enum@[{\@ifnextchar<{\beamer@enumdefault[}{\beamer@@@enum@[}}
\def\beamer@enumdefault[#1]{\def\beamer@defaultospec{#1}%
  \@ifnextchar[{\beamer@@@enum@}{\beamer@enum@}}
\def\beamer@@@enum@[#1]{% partly copied from enumerate.sty
  \@enLab{}\let\@enThe\@enQmark
  \@enloop#1\@enum@
  \ifx\@enThe\@enQmark\@warning{The counter will not be printed.%
    ^^J\space\@spaces\@spaces\@spaces The label is: \the\@enLab}\fi
  \def\insertenumlabel{\the\@enLab}
  \def\beamer@enumtempl{enumerate mini template}%
  \expandafter\let\csname the\@enumctr\endcsname\@enThe
  \csname c@\@enumctr\endcsname7
  \expandafter\settowidth
  \csname leftmargin\romannumeral\@enumdepth\endcsname
  {\the\@enLab\hspace{\labelsep}}%
  \beamer@enum@}
\def\beamer@enum@{%
  \beamer@computepref\@itemdepth% sets \beameritemnestingprefix
  \usebeamerfont{itemize/enumerate \beameritemnestingprefix body}%
  \usebeamercolor[fg]{itemize/enumerate \beameritemnestingprefix body}%
  \usebeamertemplate{itemize/enumerate \beameritemnestingprefix body begin}%
  \expandafter
  \list
  {\usebeamertemplate{\beamer@enumtempl}}
  {\usecounter\@enumctr%
    \def\makelabel##1{{\hss\llap{{%
            \usebeamerfont*{enumerate \beameritemnestingprefix item}%
            \usebeamercolor[fg]{enumerate \beameritemnestingprefix item}##1}}}}}%
  \setlength\itemsep{\fill}
  \ifnum \@itemdepth >1
  \vfill
  \fi%
  \beamer@cramped%
  \raggedright%
  \beamer@firstlineitemizeunskip%
}
\def\endenumerate{\ifhmode\unskip\fi\endlist%
  \usebeamertemplate{itemize/enumerate \beameritemnestingprefix body end}
  \ifnum \@itemdepth >1
  \vfil
  \fi%
}
\makeatother

% Section title page
\AtBeginSection[]{
  \begin{frame}
  \vfill
  \centering
  \begin{beamercolorbox}[sep=8pt, center, shadow=true, rounded=true]{title}
    \usebeamerfont{title}\insertsectionhead\par%
  \end{beamercolorbox}
  \vfill
  \end{frame}
}

% Item icon
% Itemize
\setbeamertemplate{enumerate item}[square]
\setbeamertemplate{itemize item}{\raise0.1ex \hbox{\scriptsize\faAtom}}
\setbeamertemplate{itemize subitem}{\raise0.15ex \hbox{\tiny\faDiceD20}}
\setbeamertemplate{itemize subsubitem}{\raise0.1ex \hbox{\tiny\faCog}}

% -------------
% Maths support
% -------------
% Load maths packages
%% Typical packages
\usepackage{%
  amsmath,%
  amsthm,%
  amssymb,%
  mathrsfs,%
  mathtools,%
  gensymb,%
  braket,%
  mleftright,%
  nicefrac,%
  interval,%
  empheq,% for boxing aligned equations
  stmaryrd,% for double-lined brackets
  accents,% for accent fine-tuning
}

% Settings for packages
%% interval: change open fences to round brackets
\intervalconfig{soft open fences}

%% mleftright: redefine \left as \mleft and \right as \mright
\mleftright


% ------------------
% Scientific support
% ------------------
% Chemistry
\usepackage[version=4]{mhchem}

% Scientific typesetting
\usepackage{siunitx}


% ----------
% Typography
% ----------
% Context-sensitive quotation marks
\usepackage[english=british]{csquotes}

%% Micro-typography
%\usepackage[%
%  activate={true, nocompatibility},%
%  draft,%
%  tracking=true,%
%  factor=1100,%
%%  stretch=10,%
%%  shrink=10%
%]{microtype}
%
%% Reduce spacing between sc characters
%\SetTracking{encoding={*}, shape=sc}{40}


% -------------------
% Font configurations
% -------------------
% Font support
\usepackage[no-math]{fontspec}
\defaultfontfeatures{%
  Ligatures=TeX,%
  Numbers=Lining,%
}

\newfontfamily\Raleway{Raleway}[%
  UprightFont=*-Regular,%
  ItalicFont=*-Italic,%
  BoldFont=*-Bold,%
  BoldItalicFont=*-BoldItalic,%
  Ligatures=TeX,%
]
\newfontfamily\Lato{Lato}[%
  Ligatures=TeX,%
]

\usefonttheme{structurebold}

% Main font selection

\setmainfont{Lato}[%
  UprightFont=*-Regular,%
  ItalicFont=*-Italic,%
  BoldFont=*-Bold,%
  BoldItalicFont=*-BoldItalic,%
  Ligatures=TeX,%
]
\setsansfont{Lato}[%
  UprightFont=*-Regular,%
  ItalicFont=*-Italic,%
  BoldFont=*-Bold,%
  BoldItalicFont=*-BoldItalic,%
  Ligatures=TeX,%
]

\setbeamerfont{headline}{family=\Raleway}
\setbeamerfont{headline title}{size=\huge, series=\bfseries}
\setbeamerfont{headline author}{size=\Large}
\setbeamerfont{headline institute}{size=\normalsize}
\setbeamerfont{footline}{family=\Raleway}
\setbeamerfont{footline address}{size=\normalsize}
\setbeamerfont{block title}{family=\Raleway, size=\LARGE, series=\bfseries}
\setbeamerfont{block body}{size=\large}
\setbeamerfont{heading}{family=\Lato, series=\bfseries}
\setbeamerfont{caption}{size=\small}

\usepackage{fontawesome5}

% Maths font selection
\usepackage[%
  math-style=ISO,%
  bold-style=ISO,%
  warnings-off={mathtools-colon},%
]{unicode-math}

% Main maths font
\setmathfont{Fira Math}

% Missing glyph ranges
\setmathfont{STIX Two Math}[%
  range={cal, bfcal, sfup, frak, bffrak},%
]
\setmathfont{STIX Two Math}[%
  range={scr, bfscr}, StylisticSet=1,%
]

% ----------------
% Language support
% ----------------
% polyglossia requires fontspec
\usepackage{polyglossia}
\setmainlanguage[variant=british]{english}

% Date and time format
\usepackage{datetime}
\newdateformat{monthyeardate}{%
  \monthname[\THEMONTH] \THEYEAR
}


% -------
% Colours
% -------
% Colour aliases
\colorlet{SpinColour}{DarkGreen}
\colorlet{SpatialColour}{DeepPink3}


% ------
% Tables
% ------
\usepackage{booktabs, multirow}


% --------
% Diagrams
% --------
% Graphics
\usepackage{graphicx, transparent}
\usepackage[export]{adjustbox}

% TikZ & PGFPlots
\usepackage{pgfplots}

\pgfplotsset{compat=1.17}

\usetikzlibrary{%
  external,%
  calc,%
  luamath,%
  positioning,%
  decorations.pathreplacing,%
  arrows.meta,%
  3d,%
  backgrounds,%
  overlay-beamer-styles,%
  patterns,%
  spy,%
}

\usepgfplotslibrary{%
  groupplots,%
  patchplots,%
  colormaps,%
  fillbetween,%
}

% PGFPlots cycle colours
\pgfplotscreateplotcyclelist{coloronly}{%
  {red},%
  {blue},%
  {black!60!green},%
  {black!20!orange},%
  {green!30!brown},%
  {blue!40!red},%
  {black!60!blue},%
  {black!40!yellow},%
  {red!50!pink},%
  {green!70!blue},%
}

% TikZ styles
\tikzset{
  nodeStyleGreen/.style={
    draw=Tikzgreen,
    fill=Tikzgreenlight,
    align=left,
    thick,
    rounded corners
  }
}

\tikzset{
  nodeStyleRed/.style={
    draw=Tikzred,
    fill=Tikzredlight,
    align=left,\right)
    thick,
    rounded corners
  }
}

\tikzset{
  nodeStyleBlue/.style={
    draw=Tikzblue,
    fill=Tikzbluelight,
    align=left,
    thick,
    rounded corners
  }
}

\tikzset{
  lineStyleRed/.style={
    color=Tikzred,
    opacity=0.75,
    line width=2pt,
  }
}

\tikzset{
  lineStyleGreen/.style={
    color=Tikzgreen,
    opacity=0.75,
    line width=2pt,
  }
}

\tikzset{
  lineStyleBlue/.style={
    color=Tikzblue,
    opacity=0.75,
    line width=2pt,
  }
}

\def\MarkLt{6pt}
\def\MarkSep{3pt}
\tikzset{
  TwoMarks/.style={
    postaction={
      decorate,
      decoration={
        markings,
        mark=at position #1 with{
          \begin{scope}[xslant=0.2]
            \draw[line width=\MarkSep,white,-] (0pt,-\MarkLt) -- (0pt,\MarkLt);
            \draw[-] (-0.5*\MarkSep,-\MarkLt) -- (-0.5*\MarkSep,\MarkLt);
            \draw[-] (0.5*\MarkSep,-\MarkLt) -- (0.5*\MarkSep,\MarkLt);
          \end{scope}
        }
      }
    }
  },
  TwoMarks/.default={0.5}
}

% Plot conditionals
\usepackage{xifthen}

% Equation highlighting
\usepackage[customcolors, beamer, markings]{hf-tikz}

% TikZ externalize switch
\newif\iftikzex
%\tikzextrue
\tikzexfalse

% Command for tikzexternalize/includegraphics
\tikzexternalize
\makeatletter
\newcommand*{\useexternalfile}[4]{
  \iftikzex
    \tikzsetnextfilename{tikzoutput/#4-output}
    \scalebox{#1}{\input{\tikzexternal@filenameprefix#4.tikz.tex}}
  \else
    \includegraphics[scale=#1, trim=#2 0 #3 0]{\tikzexternal@filenameprefix tikzoutput/#4-output.pdf}
  \fi
}
\makeatother
\tikzexternaldisable

% Subplots
\usepackage{subcaption}


% ----------
% Animations
% ----------
\usepackage[final, controls, buttonsize=0.8em]{animate}


% ------------------------
% References & bookmarking
% ------------------------
% Hyperlink references
\usepackage{hyperref}

% Footnotes
\renewcommand{\thefootnote}{\alph{footnote}}

% Footnote kerning with punctuation
% Do not use with option multiple for footmisc since this package has a different way of handling multiple footnotes.
\usepackage{fnpct}

% Bibliography
\usepackage[
  sorting=none,
  giveninits=true,
  style=nature,
  autocite=superscript,
  dateabbrev=false,
  doi=false,
  url=false,
  isbn=false,
  backend=biber
]{biblatex}
\addbibresource{bib/magic-sep2022.bib}

\newrobustcmd*{\footlessfullcite}{
  \AtNextCite{
    \renewbibmacro{title}{}\renewbibmacro{in:}{}
  }
  \footfullcite
}
\makeatletter
  \def\@makefnmark{}
\makeatletter
\AtEveryCitekey{\iffootnote{\tiny}{\footnotesize}}

\renewbibmacro*{name:andothers}{% Based on name:andothers from biblatex.def
  \ifboolexpr{
    test {\ifnumequal{\value{listcount}}{\value{liststop}}}
    and
    test \ifmorenames
  }
  {\ifnumgreater{\value{liststop}}{1}
     {\finalandcomma}
     {}%
   \andothersdelim\bibstring[\textit]{andothers}}
  {}%
}

\setlength{\footnotesep}{2pt}
\addtobeamertemplate{footnote}{\hskip -1.7em}{}


% -----------------
% Cross-referencing
% -----------------
% To be loaded after hyperref
\usepackage[capitalise, noabbrev]{cleveref}


% --------------------------
% Special symbols & commands
% --------------------------
% Symbols
\newcommand*{\D}{\mathrm{d}}
\newcommand*{\euler}{e}
\newcommand*{\imag}{i}
\newcommand*{\T}{\symsfup{T}}

% Commands
\newcommand*{\vecop}[1]{\hat{\boldsymbol{#1}}}
\newcommand*{\norm}[1]{\left\lVert#1\right\rVert}
\newcommand*{\emphbf}[2]{%
  \textbf{\color{#1}#2}%
}
\newcommand*{\emphit}[2]{%
  \textit{\color{#1} #2}%
}
\newcommand*\widefbox[1]{\fbox{\hspace{1em}#1\hspace{1em}}}

% Operators
\DeclareMathOperator{\Tr}{tr}
\DeclareMathOperator{\Ln}{Ln}
\DeclareMathOperator{\id}{id}
\DeclareMathOperator{\diag}{diag}
\DeclareMathOperator*{\argmin}{argmin}
\DeclareMathOperator*{\re}{\symfrak{Re}}


%% Latin abbreviations
\usepackage{xspace}
\newcommand*{\eg}{\textit{e.g.}\@\xspace}
\newcommand*{\ie}{\textit{i.e.}\@\xspace}
\newcommand*{\ca}{\textit{ca.}\@\xspace}
\newcommand*{\cf}{\textit{cf.}\@\xspace}

\makeatletter
\newcommand*{\etc}{%
    \@ifnextchar{.}%
        {\textit{etc}}%
        {\textit{etc}.\@\xspace}%
}
\makeatother

% New commands
\makeatletter
\newcommand{\annotate}[5]{
  % Format: \annotate{overlay scope}{colour}{path}{alignment}{text}
  \begin{tikzpicture}[overlay]
    \draw<#1>[stealth-, #2, draw, use marker id] #3 node[#4]{#5};
  \end{tikzpicture}
}

\newcommand*\bigcdot{
  \mathpalette\bigcdot@{.5}
}
\newcommand*\bigcdot@[2]{
  \mathbin{
    \vcenter{
      \hbox{\scalebox{#2}{$\m@th#1\bullet$}}
    }
  }
}
\makeatother

% List of symbols
\usepackage[%
  nonumberlist,%
  symbols,%
  nogroupskip,%
  nopostdot,%
  nohypertypes={symbols}
]{glossaries-extra} % package for glossaries


% Hyphenated long forms
\glsaddkey
  {hyphenated}        % new key
  {\relax}            % default value if "hyphenated" isn't used in \newglossaryentry
  {\glsentryhyphx}    % analogous to \glsentrytext
  {\Glsentryhyphx}    % analogous to \Glsentrytext
  {\glshyphx}         % analogous to \glstext
  {\Glshyphx}         % analogous to \Glstext
  {\GLShyphx}         % analogous to \GLStext
\newcommand{\GENglspostlinkhook}{%
  \ifglsused{\glslabel}{}{ (\glsentryshort{\glslabel})}\glsunset \glslabel}
\makeatletter
\newcommand\metadef[1]{%
  \expandafter\newcommand\csname gls#1\endcsname{%
    \@ifstar{\csname sgls#1\endcsname}{\csname ngls#1\endcsname}%
  }
  \@namedef{sgls#1}##1{{\let\glspostlinkhook \GENglspostlinkhook\expandafter\csname gls#1x\endcsname*{##1}}}%
  \@namedef{ngls#1}##1{{\let\glspostlinkhook \GENglspostlinkhook\expandafter\csname gls#1x\endcsname{##1}}}%
  \expandafter\newcommand\csname Gls#1\endcsname{%
    \@ifstar{\csname sGls#1\endcsname}{\csname nGls#1\endcsname}%
  }
  \@namedef{sGls#1}##1{{\let\glspostlinkhook \GENglspostlinkhook\expandafter\csname Gls#1x\endcsname*{##1}}}%
  \@namedef{nGls#1}##1{{\let\glspostlinkhook \GENglspostlinkhook\expandafter\csname Gls#1x\endcsname{##1}}}%
  \expandafter\newcommand\csname GLS#1\endcsname{%
    \@ifstar{\csname sGLS#1\endcsname}{\csname nGLS#1\endcsname}%
  }
  \@namedef{sGLS#1}##1{{\let\glspostlinkhook \GENglspostlinkhook\expandafter\csname GLS#1x\endcsname*{##1}}}%
  \@namedef{nGLS#1}##1{{\let\glspostlinkhook \GENglspostlinkhook\expandafter\csname GLS#1x\endcsname{##1}}}%
}
\makeatother

\metadef{hyph}

\loadglsentries{symbols/symbols}
\loadglsentries{symbols/acronyms}

\makeatletter
% Thanks to Paul Gaborit https://tex.stackexchange.com/a/179946/38080
\def\extractcoord#1#2#3{
  \path let \p1=(#3) in \pgfextra{
    \pgfmathsetmacro#1{\x{1}/\pgf@xx}
    \pgfmathsetmacro#2{\y{1}/\pgf@yy}
    \xdef#1{#1} \xdef#2{#2}
  };
}
\makeatother
\newcommand{\showboundingbox}{%
    %
    % Show the bounding box of the tikzpicture. Use as last command.
    % It will not change the bounding box thanks to the overlay option.
    %
    \extractcoord\xa\ya{current bounding box.south west}
    \extractcoord\xb\yb{current bounding box.north east}
    \node [overlay, draw=red, fill=white, opacity=0.8, font=\tiny\ttfamily, anchor=north east] at (current bounding box.north east) {(\xa, \ya) (\xb, \yb)};
}